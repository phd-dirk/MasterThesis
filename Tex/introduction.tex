\chapter{Introduction}
The pertubative expansion in QCD is known to lead to a divergent series, which is at best asymptotic. The asymptotic behavior of the pertubative series manifest itself in the appearance of singularities for its Borel transform. Those singularities connected with renormalization are termed renormalons.
\par
The presence of infrared (IR) renormalons, on the positive real Borel axis, lead to ambiguities. They appear due to a splitting, caused by the operator product expansion (OPE) of the function into an pertubative and and operator correction part. The splitting contains ambiguities, which should cancel, because the full function cannot have ambiguities in its definitions to be a physical quantity. In the framework of the OPE higher dimensional operators corrections appear, which contain the associated ambiguities, such that the full function is unambiguous. Those operators are the so called QCD condensates. The operators that display renormalon ambiguities are a subset of those that arise in the framework of the OPE.
\par 
Limiting ourselves to correlation functions of vector or axialvector currents with respect to the QCD vacuum, the lowest-lying IR renormalon pole is associated to the vacuum matrix element of one dimension-4 operator, the gluon condensate. The next-closest singularity then is found to correspond to the dimension-6 triple gluon condensate and a set of dimension-6 4-quark condensates. It is these latter 4-quark condensates that we intend to investigate in more detail in the present work.
\par
In the upcoming chapter we want to provide the fundamentals of renormalization in QCD. Our main issue will be to present one-loop calculations of anomalous-dimension matrices $\gamma$ for four-quark operators. In addition to the computations of the anomalous dimension matrix we want to give the basics of the renormalisation group equation (RGE) and test it for the from us calculated anomalous dimension matrix.
